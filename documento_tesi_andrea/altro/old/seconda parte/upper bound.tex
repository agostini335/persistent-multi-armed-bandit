\documentclass{article}
\usepackage[utf8]{inputenc}
\usepackage{times}  %Required
\usepackage{helvet}  %Required
\usepackage{courier}  %Required
\usepackage{url}  %Required
\usepackage{graphicx}  %Required

\usepackage{amssymb}
\usepackage{amsmath}
\usepackage{tikz}
\usepackage{pgf}
\usetikzlibrary{calc}
\usepackage{algorithm}
\usepackage{ntheorem}\usepackage{amsmath}
\usepackage{amsfonts}
\usepackage{amssymb}
\usepackage[noend]{algpseudocode}
\usepackage{thm-restate}
\usepackage{todonotes}
\usepackage{enumitem}
\usepackage{pgfplots}
\usepackage{mathtools}
\usepackage{bm}

\frenchspacing  %Required
\setlength{\pdfpagewidth}{8.5in}  %Required
\setlength{\pdfpageheight}{11in}  %Required

\newtheorem{theorem}{Theorem}
\newtheorem{proof}{Proof}%[theorem]
\newtheorem{corollary}{Corollary}[theorem]
\newtheorem{lemma}[theorem]{Lemma}
\newtheorem{definition}{Definition}
\newtheorem{example}{Example}
%\newtheorem{problem}{Problem}
\newtheorem{prop}{Proposition}
\newtheorem{observation}{Observation}
\newtheorem{property}{Property}
\newtheorem{remark}{Remark}
\DeclareMathOperator*{\argmin}{arg\,min}
\DeclareMathOperator*{\argmax}{arg\,max}
\DeclareMathOperator*{\pl}{\mathcal{P}}
\DeclareMathOperator*{\pre}{\sqsubseteq}
\DeclareMathOperator*{\I}{\mathcal{I}}
\DeclareMathOperator*{\T}{\mathcal{T}}
\DeclareMathOperator*{\F}{\mathcal{F}}
\DeclareMathOperator*{\B}{\mathcal{B}}

\title{Upper Bound}
\author{}
\date{October 2018}


\begin{document}

\noindent
$\textbf{Best-Response problem in sequence form}$
The best-response problem of player $i$ against strategy $r_adv$ of the adversarial player in sequence form is formulated as: \\




\begin{equation}
\max_{r_1} \sum_{q_1 \in Q_1} \sum_{q_{adv} \in Q_{adv}} r_1^T(q_1) U_1(q_1,q_{adv}) r_{adv}^T(q_{adv}) 
\end{equation}



\begin{algorithm}[!h]
	\caption{\texttt{Upper Bound}}
	\begin{scriptsize}
		\begin{algorithmic}

		    \Function{UB}{$\Gamma$, $h$, $\bm{\pi}^{adv}$, $\bm{\pi}^1$}
		    % devo dire che il primo h passato è il nodo radice
		    
		    \If{$h \in Z$}  
		    \State\textbf{return} $u(h)$
		    \EndIf
		    
		    \State $v(h) \leftarrow -\infty$
		    \If{$(h \in I | P(I)=1)$}
		    
		    \For {$a\in A(h)$}
		    \State $v(h)\leftarrow max\{v(h)$, $\sum_{a \in A(I)} {\pi}_a^1$ UB$(\Gamma$, $ha$, $\bm{\pi}^{adv}$, $\bm{\pi}^1) \}$
		    \EndFor
		    \EndIf
		    
		    \If{$(h \in I | P(I)=adv)$}
		    \For {$a\in A(h)$}
		    \State $v(h)\leftarrow max\{v(h)$, $\sum_{a \in A(I)} {\pi}_a^{adv}$ UB$(\Gamma$, $ha$, $\bm{\pi}^{adv}$, $\bm{\pi}^1) \}$
		    \EndFor
		    \EndIf
		    
		    \State\textbf{return} $(\Gamma,\textbf{v})$
		    \EndFunction
			
		\end{algorithmic}
	\end{scriptsize}
	\label{alg:UpperBound}
\end{algorithm}



\end{document}
