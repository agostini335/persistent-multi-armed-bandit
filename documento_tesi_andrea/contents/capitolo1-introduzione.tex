\chapter{Introduction}


\section{Research Area and Main Problem}

%-problema:processo in cui ci sono da prendere delle decisioni sequenziali 

Sequential decision making under uncertainty is one of the most important challenges within the research field of artificial intelligence, as it describes a wide variety of real-world scenarios. In many everyday situations, an \emph{agent}, or a decision maker, has to choose between alternatives to achieve his/her goals. These situations vary from simple daily routines, such as go to work or watch a movie, to complex problems, such as financial investment or web content optimization. In particular, in the former, a person has to decide how to reach the workplace, and therefore, everyday he/she will choose whether to take the car or go by public transport. On the other hand, the latter consists in selecting the best items to display for a given user visit (i.e., page view) from a set of available options. In any case, what makes decision making such a difficult task is the \emph{uncertainty} of the outcome of a decision. Furthermore, the outcome of a decision is usually affected by external factors unknown by the agent.
For example, in some circumstances, driving to work is generally quicker than taking the train, but the state of traffic could easily invert the situation. In the same way, a particular web content could drastically change its attractiveness due to an abrupt change in social trends. The outcome of a decision, which is typically a \emph{reward}, is revealed to the agent only after the decision has been taken. This will lead the agent to take better choices thanks to the acquired knowledge.
-il goal dell'agente è massimizzare il reward
-In studying the sequential problems described above it is usually assumed that the reward acquired after an action taken by the agent is a real number provided to the agent immediately after the action has taken place. Nei casi visti il reward potrebbe esprimere il tempo trascorso per completare il viaggio casa lavoro o il numero di CTK che un determinato contenuto web ha ottenuto. In many application this is a limitation, infact if we consider......
-subscription prezzi alti/bassi
-clinical trial criticità dei casi attuali
-recsys
- Un natuale adattamento farebbe cadere il problema sotto l'area decision making under delayed feedback, we want to exploit also the partial information visibile by the agent during the reward process. With persistent we refer

This thesis studies sequential decision making for types of situations which involve persistent rewards. We design and study algorithm for one of the most important  ..... abstraction of deci This is called the Multi-Armed Bandit (MAB) problem.



%esempi

%standard case: reward istantanee che seguono azioni prese in ogni istante temporale

%noi ci colochiamo nel caso particolare in cui la reward è distribuita nel tempo che segue l'azione

%(spiegazione termine persistent)

%esempi: subscription 

%scopo: in questo setting vogliamo sfruttare reward parziale che abbiamo nel tempo

%quindi proponiamo degli algoritmi x intgrare questa info mentre arriva




\section{Original Contributions}
We start by focusing on s $\textsf{P} = \textsf{NP}$, even on a completely inflated tree.
Then, we introduce the \emph{hybrid column generation} algorithm which computes optimal team-maxmin equilibria with coordination devices~\citep{celli18}. We can use this 


\section{Thesis Structure}

The Thesis is structured in the following way:
\begin{itemize}
    \item In Chapter \ref{C6} we draw conclusions. We summarize the main results of our work and we propose some future developments.
\end{itemize}




